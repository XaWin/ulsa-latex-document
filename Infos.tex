% --- --- --- --- --- --- --- --- --- --- --- --- --- --- %
%   Paramètrage du titre & des informations du document   %
% --- --- --- --- --- --- --- --- --- --- --- --- --- --- %
% La matière se définit comme la période durant laquelle
% Exemple : un examen de Juin 2016 reçu en 2018 serait


%       1. MODULE
% Choix du module, format mxx (en minuscule, avec zéro si besoin)
\setUlsaModule{m00}


%       2. SEMESTRE RELATIF A LA MATIERE DU DOCUMENT
% 1er champ : choix du semestre (SA) ou (SP)
% 2ème champ : choix de l'année de référence, en format 4 chiffre
\setUlsaSemester{SP}{2018}


%       3. TITRE DU DOCUMENT
\setUlsaTitle{Exemplaire de travail propédeutique}


%       3b. SOUS-TITRE DU DOCUMENT (Optionel)
\setUlsaSubtitle{Évaluation et corrections du travail portant l'action en paiement non chiffrée (CPC 85)}

%       4. AUTEUR(S)
% Un auteur est une personne ayant rédigé le texte
\addAuthor{Xavier}{Winterhalter}


%       5. EDITEUR(S) (optionel)
% Un éditeur est une personne ayant préparé le texte, par exemple dans LaTeX
% \addEditor{Prénom}{Nom}


%       6. DATE ET LIEU DE PRODUCTION (optionel)
% Lieu par défaut: Brig // Date par défaut: le jour même
%\setUlsaDate{21 août 2018}
\setUlsaLocation{Zürich}