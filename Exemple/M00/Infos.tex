% --- --- --- --- --- --- --- --- --- --- --- --- %
%   Modèle de titre pour le M00 - Propédeutique   %
% --- --- --- --- --- --- --- --- --- --- --- --- %

%       1. SEMESTRE QUAND LE TRAVAIL A ETE RENDU
% Comme indiqué dans le titre
% 1er champ : choix du semestre (SA) ou (SP)
% 2ème champ : choix de l'année de référence, en format 4 chiffre
\setUlsaSemester{SP}{2018}


%       2. SOUS-TITRE DU DOCUMENT (Optionel)
% Comme indiqué dans le titre, par exemple : le séquestre (art. 263 ss CPP)
\setUlsaSubtitle{Évaluation et corrections du travail portant sur\\LE SUJET}

%       3. AUTEUR(S)
% Anonyme, laissé vide + caviarder les PDFs ou le nom de l'étudiant
% \addAuthor{Prénom}{Nom}



% --- --- --- --- --- --- --- --- --- --- --- --- %
%   Modèle de titre pour le M00 - Propédeutique   %
% --- --- --- --- --- --- --- --- --- --- --- --- %

% MODULE
% Choix du module, format mxx (en minuscule, avec zéro si besoin)
\setUlsaModule{m00}

% TITRE DU DOCUMENT
% Titre principale fix
\setUlsaTitle{Exemplaire de travail propédeutique}

% EDITEUR(S) 
% Sans éditeur, paramètre par défaut (ULSA)

% DATE ET LIEU DE PRODUCTION
% Date et publication par défaut

