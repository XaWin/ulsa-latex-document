\subsubsection{Casuistique de la qualification des contrats}

\paragraph{Mandat}
Contrat par défaut : si un contrat innomé contient des clauses visant un travail, alors le mandat s'applique (CO 394 II)

Contrat d'expertise (arrêt 34.1) : si l'expertise est une expertise qui ne peut pas être jugée selon des critères objectifs et dont l’expert ne peut pas garantir l’exactitude objective du résultat, il s'agit d'un mandat (sinon entreprise).

\paragraph{Contrat de travail}

\paragraph{Contrat d'entreprise}
Contrat d'expertise (arrêt 34.1) : si l'expertise est une "expertise technique", qui "aboutit normalement à un résultat susceptible d’être vérifié selon des critères objectifs et qualifié de juste ou d’erroné", alors c'est un contrat d'entreprise (sinon mandat).