Les contrats peuvent être soit nommés s'ils sont spécifiquement réglés par la loi (ex. de bail ou la vente.). A l'opposé, il y a les contrats \textit{innomés} qui peuvent se développer en deux sous-classes.

Les \textbf{\textit{contrats mixtes}} sont des contrats qui réunissent en un même accord des prestations appartenant à des contrats nommés différents. Les contrats \textit{\textbf{sui generis}} sont des contrats dont le contenu n'est visé par aucune forme de contrat. Certains de ces contrats sui generis sont tellement acceptés dans la vie courante qu'ils en deviennent des \textit{contrats typiques commerciaux}.

\textbf{Contrats \textit{sui generis} couverts dans le cours}

\begin{itemize}
    \item Contrat de restauration, arrêt 101
    \item Contrat de remise de commerce, arrêt 22
    \item Contrat de représentation exclusive, arrêt 172
    \item Contrat d’abonnement téléphonique, arrêt 6.4, 71
\end{itemize}

\textbf{Contrat de restauration, arrêt 101}
L'aubergiste qui reçoit un consommateur dans son établissement conclut avec lui un contrat sui generis qui l'oblige non seulement a lui offrir contre espèces des boissons et aliments de qualité correspondante, mais a les lui laisser consommer sur place, sans qu'il en résulte un préjudice pour sa santé ou son intégrité corporelle, et cette obligation n'est pas accessoire, elle est principale au même titre que les autres.